\documentclass[a4paper,11pt]{article}
\usepackage[a4paper,total={18cm, 24cm}]{geometry}
\usepackage[parfill]{parskip}
\usepackage[utf8]{inputenc}
\usepackage[T1]{fontenc}
\usepackage{fancyhdr}
\usepackage[ddmmyyyy]{datetime}
\usepackage{graphicx}
\usepackage{subcaption}
\usepackage{multirow}
\usepackage{hyperref}
\usepackage{amsfonts}

\pagestyle{fancy}
\fancyhf{}
\lhead{\today}
\chead{QUA - HW2: Quantum simulator}
\rhead{Jakub Rada}

\begin{document}
I tried to implement the Quantum Circuits model for quantum computing and tested it on the GHZ state.
I also checked my results agains GHZ in the Qiskit framework.

My goal was to try to implement it in a very general way to allow easier addition of more complicated gates and operations.
I describe it more thoroughly in the README file.

\subsection*{GHZ}
There is a prepared a target for the GHZ state in a file \texttt{src/ghz.cpp}.
It can be compiled using CMake and run with the following steps:
\begin{enumerate}
    \item \texttt{mkdir build \&\& cd build}
    \item \texttt{cmake ..}
    \item \texttt{make ghz}
    \item \texttt{./ghz}
\end{enumerate}
The program prepares the quantum circuit for preparing the GHZ qubit and measuring all three qubits, each into its own classical register.
It is set to work on a three qubit system and $1024$ shots for measurement (as is the default in qiskit, I believe).

The results for three consecutive experiments look as follows:

\begin{table}[ht]
    \centering
    \begin{tabular}{| c | c | c | c |}
        \hline
        outcomes & counts (exp 1) & counts (exp 2) & counts (exp 3) \\
        \hline
        \hline
        $000$    & $506$          & $534$          & $500$          \\
        \hline
        $111$    & $518$          & $490$          & $524$          \\
        \hline
        \hline
        sum      & $1024$         & $1024$         & $1024$         \\
        \hline
    \end{tabular}
\end{table}

More details about the implementation and API of the simulator is provided in the README.md file.

\end{document}
